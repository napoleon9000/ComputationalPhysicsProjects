\documentclass{article}
\usepackage{listings}
\usepackage{graphicx}
\usepackage{amsmath}

\begin{document}
\lstset{language=Pascal}          % Set your language (you can change the language for each code-block optionally)	
\title{Report of ...}
\author{Jialin Liu}
\date{\today}
\maketitle

\newpage

\section{Introduction}
\section{Theoretical models and technicalities}
\subsection{Mathematical model}	
To solve ordinary differential equation, Euler method is commonly used to achieve $o(h62)$ accuracy as shown in Eq.1:\\

\begin{align}
\frac{x_{i+1}+x_{i-1}-2*x_i}{h^2} = f{''}_i,
\end{align}

where h is the step length, $x(i)$ is the $i^th$ points and $f$ is the function value at the $i^th$ point. \\
This method can be implemented to solve many ODEs involving first and second order derivatives in the following form numerically:\\

\begin{align}
\frac{d^2y}{dx^2}+k^2(x)y = f(x),
\end{align}

where $k^2$ is a real function.\\
For example, for the Poission's equation under spherical symmetrical field using polar coordinations, the original equation can be simplified as following form:\\

\begin{align}
\frac{1}{r^2}\frac{d}{dr}\left(r^2\frac{d\Phi}{dr}\right) = -4\pi \rho(r),
\end{align}

where $\phi$ is the electrostatic potential, $\rho(r)$ is the local charge distribution and $r$ is the radial distance\\
If we substitute $\Phi(r)= \phi(r)/r$, we can have \\

\begin{align}
\frac{d^2\phi}{dr^2}= -4\pi r\rho(r).
\end{align}




\newpage
\begin{figure}[h]
	\centering
	\includegraphics[width=1.2\linewidth]{C:/Users/jlliu/Desktop/Picture1}
	\caption{}
	\label{fig:Picture1}
\end{figure}

$\sum_i^k{a+b}$ 

\begin{align*}
a+b
\end{align*}

\centering{
	\begin{tabular}{|l|r|r|}
		\hline
		a&b&c\\
		\hline
		e&f&g\\
		\hline
	\end{tabular}
} 


\begin{verbatim}
for i in range(1, 5):
print i
else:
print "The for loop is over"
\end{verbatim}



\begin{lstlisting}[frame=single]  
for i in range(1, 5):
print i
else:
print "The for loop is over"
\end{lstlisting}


\end{document}
